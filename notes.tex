\documentclass[letterpaper]{article}
\usepackage{pgffor}
\usepackage{geometry}
\usepackage{listings}
\usepackage[section]{minted}
\usepackage{python}
\usepackage{hyperref}
\usepackage{siunitx}
\usepackage{multicol}

\geometry{
	letterpaper,
	left=20mm,
	right=20mm
}

\sisetup{output-exponent-marker=\ensuremath{\mathrm{e}}}
\sisetup{group-separator = {.}}
\setlength{\columnsep}{1cm}

\newmintedfile[phpcode]{php}{
bgcolor=mintedbackground,
fontfamily=tt,
linenos=true,
numberblanklines=true,
numbersep=5pt,
gobble=0,
frame=leftline,
framerule=0.4pt,
framesep=2mm,
funcnamehighlighting=true,
tabsize=4,
obeytabs=false, 
mathescape=false
samepage=false, %with this setting you can force the list to appear on the same page
showspaces=false,
showtabs =false,
texcl=false,
}

\title{Competitive Programming Notes}
\author{Ignacio Hermosilla, Nicholas Mc-Donnell, Javier Reyes}
\date{2018}

\begin{document}

\maketitle

\begin{python}
	import glob
	raw_files = ["." + x.lstrip('.') for x in glob.glob("./*/*/*.h")]
	raw_files += ["." + x.lstrip('.') for x in glob.glob("./*/*/*.cpp")]
	files = [x + "," for x in raw_files[:-1]] + raw_files[-1:]
	f = open("ListOfFiles.tex", "w")
	f.write("\\def\ListOfFiles{")
		for x in files: f.write(x)
		f.write("}")
\end{python}

\input{ListOfFiles}

%\setcounter{tocdepth}{1}
\tableofcontents
\newpage
\section{Notas Útiles}
\begin{multicols}{2}
	\begin{tabular}{| l | c |}
		\hline
		$O(f(n))$       & Limite                 \\
		\hline
		$O(n!)$         & $10...11$              \\
		$O(2^nn^2)$     & $15...18$              \\
		$O(2^nn)$       & $18...21$              \\
		$O(n^4)$        & \num{100}              \\
		$O(n^3)$        & \num{500}\footnotemark \\
		$O(n^2\log^2n)$ & \num{1000}             \\
		$O(n^2\log n)$  & \num{2000}             \\
		$O(n^2)$        & \num{1e4}\footnotemark \\
		$O(n\log^2n)$   & \num{3e5}              \\
		$O(n\log n)$    & \num{1e6}              \\
		$O(n)$          & \num{1e8}\footnotemark \\
		\hline
	\end{tabular}
	\addtocounter{footnote}{-3}
	\stepcounter{footnote}\footnotetext{Este caso esta justo en el limite de tiempo, además en 256 MB cabe a los una matriz de $400^3$ ints}
	\stepcounter{footnote}\footnotetext{En general solo funciona hasta \num{6e3}}
	\stepcounter{footnote}\footnotetext{En general solo funciona hasta \num{4e7}}
	\vfill
	\begin{tabular}{| l | c |}
		\hline
		\multicolumn{2}{|c|}{Primos hasta} \\
		\hline
		\num{1e2} & \num{25}               \\
		\num{1e3} & \num{168}              \\
		\num{1e4} & \num{1229}             \\
		\num{1e5} & \num{9592}             \\
		\num{1e6} & \num{78498}            \\
		\num{1e7} & \num{664579}           \\
		\num{1e8} & \num{5761455}          \\
		\num{1e9} & \num{50487534}         \\
		\hline
	\end{tabular}
\end{multicols}

El mayor HCN\footnote{Highly Compositve Number, números que tienen más divisores que cualquier número más pequeño} menor a \num{1e9} tiene \num{1344} divisores.

El ``prime gap'' hasta \num{1e9} es a lo más \num{288}


\newpage
\foreach \file in \ListOfFiles
{
	\section{\file}
	\inputminted[autogobble]{C++}{\file}
	\newpage
}


\end{document}